\chapter{Security}
\minitoc

A) the lab description of the respective topic (to make the report selfcontained)
B) 1-2 pages pointing out how much they solved and which issues they encountered
C) A “print” of an example run
D) 1-2 pages where they relate what they did to the relevant theory in the curriculum
E) max 1 page conclusion, concluding what was solved well (perhaps even makes you proud :-)- and what could be done differently/better and why\\



In distributed systems the objective for secure systems are to provide privacy, integrity and availability in the face of malicious activities like 
masquerading, tampering and denial of service. The distributed system is exposed to misuse of all sorts so the algorithms and techniques used are paramount. \\

In this chapter we explore som security topics. In section \ref{sec_solution} we present our solution to the lab exersize. In section \ref{sec_run} we demonstrate an example run of the application. In section \ref{sec_theory} we compare the solution to the underlying theory. In section \ref{sec_conclusion} we wrap up the chapter.  


\section{Lab Exersize}
\label{sec_exersize}

\textit{The main objective of this week’s assignment is to understand and analyze security models and protocols and furthermore implement a simple security protocol for task manager. As part of the assignment, you will study and develop a simple role based access control mechanism for tasks based on an authentication using ITU credentials. Moreover, you will also use one of the crypto algorithms to ensure security among all/some parts of communication.}\\


\textit{As part of the assignment, you are required to implement the updated security protocol offering the functionality mentioned above.
Analyze the protocol and identify further weaknesses and  vulnerabilities in the updated protocol. In order to do that, you may compare the updated protocol with some of the standard protocols such Kerberous etc.}

\section{Solution}
\label{sec_solution}

role based xml





\section{Example Run}
\label{sec_run}


\section{Theory}
\label{sec_theory}


\section{Conclusion}
\label{sec_conclusion}
 