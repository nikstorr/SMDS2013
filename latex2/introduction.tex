\chapter{Introduction}
\minitoc

\section{Project Presentation}

This report is an examination of some of the principles seen in the application of distributed systems. The report's modus operandi is to test the underlying theory on small independent applications. Projects in the report are based on a 'taskmanager' application. The basic concept provides a means of creating, editing, deleting and persisting tasks. \\

The report consists of 8 chapters each examining a single topic in distributed systems. The taskmanager application provides an empirical dimension, a steppingstone for the further analysis of individual topics. \\

Due to the pressure of other concurrent courses at ITU we have not been able to provide a single coherent project examining all issues, instead we use the individual projects from the course lab exercises as the background for a discussion of the topics. The projects therefore only serve as proof of concept and as a basis for analysis and discussion of concepts.  \\

In the second chapter the taskmanager application is used to set up a client-server based application. Later chapters provide additional functionality such as group communication and security, web services and serialization, etc. \\

\begin{comment}
The idea is to try out some of the techniques commonly seen in the implementation of distributed systems and thereby demonstrate our understanding of the theories behind, as well as (some of), the challenges faced by distributed systems. 
\end{comment}

\subsection{Taskmanager}

The taskmanager application is founded on an xml document containing 'tasks'. The xml document is serialized using JAXB into an object tree. Note in the google app engine project we use JPA for serializing in order to accommodate google app engine specific requirements for persistence. The taskmanager.xml file structure looks like this:
\pagebreak
\begin{lstlisting}[caption = taskmanager.xml]
<cal>
 <tasks>
  <task id="handin-01" name="Submit assignment-01" date="16-12-2013"
	status="not-executed" required="false">
	<attendants>student-01, student-02</attendants>
	<role>student</role>	
  </task>
  - 
  - 
  - 
 </tasks>
</cal>
\end{lstlisting}

\subsection{Task}
The taskmanager application holds a list of Task objects. A task object has the following properties:\\

\begin{comment}
ID (a unique identifier), name (the name of the task), date (the date of creation), status (status of the task. E.g executed or non-executed.),
required (E.g. ?true? or ?false?.?), role (role, is an access control technique in where the role of a task signifies the access rights level of that task. ), Attendants (a list of attendants for the task)\\
\end{comment}


\begin{lstlisting}[caption=Task]
@XmlRootElement(name = "task")
public class Task implements Serializable {
    @XmlID
    @XmlAttribute
    public String id;
    @XmlAttribute
    public String name;
    @XmlAttribute
    public String date;
    @XmlAttribute
    public String status;
    @XmlAttribute
    public String description;
    @XmlElement
    public String attendants;
    @XmlElement
    public String role;
    @XmlAttribute
    public Boolean required;
}

\end{lstlisting}

\section{report structure}

Chapter 1 is an introduction to the report and to some basic concepts in distributed systems. We briefly outline the definition of a distributed system, the foundations of the report. \\

In chapter 2 we discuss the java Servlet structure. A servlet is Java's answer to asp or php active server pages. They provide dynamic content between a client and a server. \\ 

In chapter 3 we create a small client-server application for communicating over TCP/IP. \\

In chapter 4 we create a RESTfull web-service. The taskmanager is made available as resources. 
Clients access task via HTTP (in the way that the HTTP protocol was originally intended to be used.) \\

In chapter 5 we discuss security in distributed systems. We implement a simple security protocol and a role-based security control mechanism and we encrypt the messages sent between nodes in the system. \\

In chapter 6 we examine indirect communication. Java's JGroups is used to establish an example to serve as the background for a discussion of the theories behind indirect/group communication. \\

In chapter 7 we look at concurrency control. We implement a simple protocol for controlling concurrent access to the taskmanager resource.\\

In chapter 8 we dive into mobile applications with android. \\

Chapter 9 concludes the report and summarizes what we've learned.\\

